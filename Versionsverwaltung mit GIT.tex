\documentclass{beamer}
\usetheme{Boadilla}
\usepackage[ngerman]{babel}
\usepackage[utf8]{inputenc}
\usepackage[T1]{fontenc}
\usepackage{lmodern}
\usepackage{hyperref}
\usepackage{listings} 
\usepackage{graphicx}
\usepackage{pgf}
\usepackage{color}
\definecolor{ex_green_titel}{RGB}{204,230,204}
\definecolor{ex_green}{RGB}{230,243,230}
\definecolor{light-gray}{gray}{0.95}

% Abschalten der Navigationsleiste!
% \beamertemplatenavigationsymbolsempty

\begin{document}
\lstset{ 						%
language=bash,                % the language of the code
basicstyle=\footnotesize,       % the size of the fonts that are used for the code
%numbers=right,                   % where to put the line-numbers
n%umberstyle=\footnotesize,      % the size of the fonts that are used for the line-numbers
%stepnumber=2,                   % the step between two line-numbers. If it's 1, each line 
                                % will be numbered
numbersep=-10pt,                  % how far the line-numbers are from the code
backgroundcolor=\color{light-gray},  % choose the background color. You must add \usepackage{color}
showspaces=false,               % show spaces adding particular underscores
showstringspaces=false,         % underline spaces within strings
showtabs=false,                 % show tabs within strings adding particular underscores
frame=single,                   % adds a frame around the code
tabsize=2,                      % sets default tabsize to 2 spaces
captionpos=b,                   % sets the caption-position to bottom
breaklines=true,                % sets automatic line breaking
breakatwhitespace=false,        % sets if automatic breaks should only happen at whitespace
%title=\lstname,                 % show the filename of files included with \lstinputlisting;
                                % also try caption instead of title
escapeinside={\%*}{*)},         % if you want to add a comment within your code
morekeywords={*,...}            % if you want to add more keywords to the set
}

\title[Einführung in Versionsverwaltung]{Eine Einführung in Versionverwaltungsysteme}   
\subtitle{Wissenswertes für GIT-Anwender}
\titlegraphic{\includegraphics[height=1.5cm]{Bilder/Logo_FAU}}
\author[Bernd Wunder]{bernd.wunder@leb.eei.uni-erlangen.de} 
\institute[LEB -- TF Uni Erlangen]{Lehrstuhl für Elektronische Bauelemente\\ Friedrich-Alexander-Universität Erlangen-Nürnberg}
\date{\today}
\logo{\includegraphics[scale=0.08]{Bilder/Logo_FAU2}}

\begin{frame}
\titlepage
\end{frame}

\begin{frame}
\frametitle{Inhaltsverzeichnis}
\tableofcontents[hidesubsections]
%\tableofcontents[currentsection,currentsubsection,hideothersubsections]
%[pausesections]
\end{frame}

%--- Einführung ---
\section{Einführung} 
\begin{frame}[c]{}
\begin{center}
\begin{Huge}
Einführung
\end{Huge}
\vspace{1.5cm}
\begin{itemize}
\item Was ist ein Versionsverwaltungssystem? 
\item Warum Versionsverwaltung? 
\item Übersicht über VCS 
\item Warum Git?
\item Grundbegriffe
\end{itemize}
\end{center}
\end{frame}

%--- Was ist ein Versionsverwaltungsystem ---
\subsection{Was ist ein Versionsverwaltungsystem}

%--- Was ist ein Versionsverwaltungsystem ---
\begin{frame}\frametitle{Was ist ein Versionsverwaltungsystem} 
\begin{quote}
Eine Versionsverwaltung ist ein System, das zur Erfassung von Änderungen an Dokumenten oder Dateien verwendet wird. Alle Versionen werden in einem Archiv mit Zeitstempel und Benutzerkennung gesichert und können später wiederhergestellt werden.
\footnote{Quelle: \href{http://de.wikipedia.org/wiki/Versionsverwaltung}{Wikipedia}}
\end{quote} 

Aufgaben
\begin{itemize}
\item  Protokollierungen der Änderungen
\item  Wiederherstellung von alten Ständen 
\item  Archivierung der einzelnen Stände eines Projektes
\item  Kooperative Entwicklung (Entwicklungsteams)
\item  gleichzeitige Entwicklung mehrerer Entwicklungszweige (branches)
\end{itemize} 
\end{frame}

%--- Warum Versionsverwaltung ---
\begin{frame}\frametitle{Warum Versionsverwaltung?}
\begin{columns}
        \column{.55\textwidth}
                \pgfimage[width=\textwidth]{Bilder/chaosschreibtisch}
        \column{.45\textwidth}
                \begin{enumerate}
                \item viele gleichzeitige laufende Projekte / Features (z.B. Sicherheitsupdates, neue Funktionstests, verschiedene Versionen)
                \item Um des Chaos Herr zu werden
                \end{enumerate}
\end{columns}
\end{frame}

%--- Warum Versionsverwaltung ---
\begin{frame}\frametitle{Warum Versionsverwaltung?}
\begin{columns}
        \column{.55\textwidth}
                \pgfimage[width=\textwidth]{Bilder/schreibtisch}
        \column{.45\textwidth}
                \begin{enumerate}
                \item viele gleichzeitige laufende Projekte / Features (z.B. Sicherheitsupdates, neue  Funktionstests, verschiedene Versionen)
                \item Um des Chaos Herr zu werden
                \item Überblick über die gesamte Entwicklung behalten
                \item kolaboratives Arbeiten in einem Team
                \end{enumerate}
\end{columns}
\end{frame}

%----- Übersicht über Versionsverwaltungsysteme -----
\subsection{Übersicht über Versionsverwaltungsysteme}

%--- Übersicht über Versionsverwaltungsysteme ---
\begin{frame}\frametitle{Übersicht über Versionssverwaltungsysteme} 
engl. Bezeichnungen: \textit{Version Control System (VCS)}, \textit{Software Configuration Management (SCM)}, \textit{Revision Control System (RCS)}
\vspace*{0.15cm}
\textbf{unvollständige Übersicht einiger VCSe:}
\begin{block}{Revision Control System (RCS)}
	\begin{tabular}{l c}
Entwicklung: & 1980 bis 2004  \\  
Einteilung: & zentrales, dateibasiertes VCS \\ 
Probleme: & Binärdateien, Verzeichnisse, locks, merge, \\
          & keine Atomic commits, keine Metadaten, umbenennen  \\
Lizenz: & GPL2 \\
Betriebsysteme: & UNIX, WIN95 \\

\end{tabular} 

\vspace*{0.3cm}
Entwickelt für die Versionsverwaltung von Text-Dateien auf einem Computer!
RCS ist im Wesentlichen mit dem ersten VCS, dem \textit{Source Code Control System (SCCS)}, vergleichbar.
\end{block}
\end{frame}

%--- Versionsverwaltungsysteme ---
\begin{frame}\frametitle{Versionsverwaltungsysteme} 
\begin{block}{Concurrent Versions System (CVS)}
	\begin{tabular}{l c}
basierend auf: & RCP (nutzt gleiches Speicherformat) \\ 
Entwicklung: & 1989 bis 2008  \\  
Einteilung: & zentrales VCS \\ 
Probleme: & Binärdateien, Verzeichnisse, locks, merge, \\
          & keine Atomic commits, keine Metadaten, umbenennen  \\
Lizenz: & GPL2 \\
Betriebsysteme: & UNIX, WINDOWS, Mac OS X \\
\end{tabular} 

\vspace*{0.3cm}
CVS wird nicht mehr aktiv weiterentwickelt. Die offizielle Webseite wird nicht mehr weiter betreut. \footnote{Quelle: \href{http://de.wikipedia.org/wiki/Concurrent_Versions_System}{Wikipedia}}
\end{block}
\end{frame}

%--- Versionsverwaltungsysteme ---
\begin{frame}\frametitle{Versionsverwaltungsysteme} 
\begin{block}{Subversion (SVN)}
	\begin{tabular}{l c}
basierend auf: & CVS (Nachfolger) \\ 
Entwicklung: & seit 2000  \\  
 & Ziele CVS Probleme (siehe oben) zu beseitigen \\
Einteilung: & zentrales VCS \\ 
Probleme: & Binärdateien, Verzeichnisse, locks, merge \\
Lizenz: & Apache License \\
Betriebsysteme: & UNIX, WINDOWS, Mac OS X \\
\end{tabular} 
\begin{quote}
Subversion versteht sich als Weiterentwicklung von CVS und entstand als Reaktion auf weit verbreitete Kritik an CVS. In der Bedienung der Kommandozeilenversion ist es sehr ähnlich gehalten.  \footnote{Quelle: \href{http://de.wikipedia.org/wiki/Subversion_\%28Software\%29}{Wikipedia}}
\end{quote}
\end{block}
\end{frame}

%--- Git ---
\begin{frame}\frametitle{Git}
\begin{columns}
        \column{.65\textwidth}
        {\small Die Entwicklung von Git wurde im April 2005 von Linus Torvalds begonnen, um das bis dahin verwendete Versionskontrollsystem BitKeeper zu ersetzen, das durch eine Lizenzänderung vielen Entwicklern den Zugang verwehrte. Die erste Version erschien bereits wenige Tage nach der Ankündigung. Derzeitiger Maintainer von Git ist Junio Hamano.}
                \begin{enumerate}
                \item Nicht-lineare Entwicklung
                \item Kein zentraler Server
                \item Datentransfer zwischen Repositories
                \item Kryptographische Sicherheit der Projektgeschichte
                \item Interoperabilität
				\item Web-Interface
                \end{enumerate}
        \column{.35\textwidth}
                \pgfimage[width=\textwidth]{Bilder/torvalds}
                
{\tiny Linus Torvalds ist Initiator von Git und des Kernels Linux}
\end{columns}
\end{frame}

%--- Vergleich einiger Versionsverwaltungsysteme ---
\begin{frame}\frametitle{Vergleich\footnote{Quelle: \href{http://better-scm.shlomifish.org/comparison/comparison.html}{Version Control System Comparison}} einiger Versionsverwaltungsysteme }
\begin{exampleblock}{Vergleich | CVS,SVN,GIT,Mercurial}
	\begin{tabular}{l|c c c c}
	\hline
		\textbf{Operation} & \textbf{CVS} & \textbf{SVN} & \textbf{GIT} & \textbf{Mercurial} \\
	\hline
	\hline
		Atomic Commits: & Nein & Ja & Ja & Ja \\
	\hline
		Umbenennen/Verschieben von\\
		Dateien und Verzeichnissen: & Nein & Ja & Ja & Ja \\
	\hline
		Intelligente Merges \\ 
		nach Umbenungen/Verschiebungen: & Nein & Nein & Nein & Ja \\
	\hline
		Verzeichnisse/Dateien mit History \\
		im Repository kopieren: & Nein & Ja & Nein & Ja \\
	\hline		
		Remote Kopie in lokales Verzeichnis: & Nein & Nein & Ja & Ja \\
	\hline		
		Änderungen an andere \\
		Repository weitergeben: & Nein & Nein & Ja & Ja \\		
	\hline		
		Changesets Support & Nein & teilw. & Ja & Ja \\ 		
	\hline
	\hline
\end{tabular} 
\end{exampleblock}
\end{frame}

%----- Warum Git? -----
\subsection{Warum Git?}

%--- Warum Git? ---
\begin{frame}\frametitle{Warum Git?}
\begin{description}
\item[Speed] Schnellstes mir bekannte Versionsverwaltungssystem 
\item[Branch] sehr einfach Zweige (Branch) zu erstellen und zusammenzuführen!
\item[Lokal] alle Informationen sind lokal gespeichert, keine Netzwerkverbindung notwendig!
\item[Speicher] sehr geringer Speicherverbrauch durch Kompression
\end{description}

\begin{figure}
\includegraphics[scale=0.25]{Bilder/speed2} 
\caption{Geschwindigkeitsvergleich von Git, Mercuial und Baszar}
\end{figure}
\end{frame}

%--- Warum Git? ---
\begin{frame}\frametitle{Warum Git?}
\begin{description}
\item[Protokoll] http, https, ssh, git, ...
\item[Kooperativ] gute Zusammenarbeit mit anderen Versionsverwaltungssystemen. Z.B. mit SVN oder CVS
\item[Stabil] sehr aktives Projekt mit vielen hundert Entwicklern
\end{description}

\begin{figure}
\includegraphics[scale=0.5]{Bilder/svnvsgit1} 
\caption{Geschwindigkeitsvergleich von Git, Mercuial und Baszar}
\end{figure}
\end{frame}

%--- Warum Git? ---
\begin{frame}\frametitle{Warum Git?}
\begin{description}
\item[tested] Protokiv im Einsatz bei Linux-Kernel (>1000 Entwickler)
\item[me] ... i know about it and i \textit{really like it} ;-)
\end{description}

\begin{figure}
\includegraphics[scale=0.2]{Bilder/kernel_Entwickler} 
\caption{Übersicht\footnote{Quelle: \href{http://www.schoenitzer.de/lks/lks.html\#new_developers}{Michael Schönitzer}} über die Anzahl der Linux Kernel Entwickler}
\end{figure}
\end{frame}

%----- Wer nutzt Git? -----
\subsection{Wer nutzt Git?}

%--- Wer nutzt Git? ---
\begin{frame}\frametitle{Wer nutzt Git?}
Viele Open Source Projekte setzen bereits Git ein: \\
( \href{https://git.wiki.kernel.org/articles/g/i/t/GitProjects_8074.html}{umfangreiche Liste auf git.wiki.kernel.org})
\begin{columns}
        \column{.5\textwidth}
                \begin{itemize}                
					\item Linux Kernel
					\item Git
					\item Android (Google's Handy OS) 
					\item CakePHP (PHP Framework) 
					\item Debian - Linux Distribution
					\item Drupal - CMS System
					\item Typo3 - CMS System
					\item Perl
					\item Ruby on Rails
	                \item VLC
                \end{itemize}
        \column{.5\textwidth}
                \begin{itemize}
                \item PostgreSQL
                \item KDE
                \item Fluxbox
                \item X.Org
                \item GCC (Gnu Compiler Collection)
                \item JQuery (JavaScript library) 
                \item Qt (Cross-platform graphic toolkit)
                \item Eclipse
                \item Gnome
                \item ...
                \end{itemize}
\end{columns}

\end{frame}

%--- Grundbegriffe ---
\subsection{Grundbegriffe}

%--- Grundbegriffe (GIT) ---
\begin{frame}\frametitle{Grundbegriffe}
\begin{block}{\textit{Repository}}
Datenbank in dem jeder Dateistand eines Projektes über die Zeit hinweg gespeichert ist.
\end{block}
\begin{block}{\textit{Working Tree}}
Arbeitsverzeichnis in dem die Modifikationen durchgeführt werden.
\end{block}
\begin{block}{\textit{Commit}}
beinhaltet alle Veränderungen bzw. spiegelt den aktuellen Zustand der in das VCS aufgenommen werden soll wieder. Enthält neben den Änderungen zusätzliche Metadaten (Commit Message, Autor, Datum, Signatur, ...)
\end{block}
\begin{block}{\textit{HEAD}}
zeigt auf die neueste Version \textit{Kopf} im aktuellen Zweig (Branch)

Achtung: Unterschide zwischen GIT und SVN,CVS
\end{block}
\end{frame}

%--- Grundbegriffe (GIT) ---
\begin{frame}\frametitle{Grundbegriffe}
\begin{block}{\textit{Secure Hash Algorithm (SHA-1)}}
ist eine eindeutige, 160 Bit (40 hexadezimale Zeichen) lange Prüfsumme für beliebige digitale Informationen.
\end{block}
\begin{exampleblock}{Beispiel:}
mit dem GNU/Linux Programm \textit{sha1sum} wird die Prüfsumme für den Text \textit{"'Isabella und Lilly Wunder"'} berechnet werden:
\lstinputlisting{bash.txt}
\end{exampleblock}
\end{frame}

%--- Grundbegriffe (GIT) ---
\begin{frame}\frametitle{Grundbegriffe (GIT)}
\begin{block}{\textit{Branch}}
bezeichnet einen parallelen Entwicklungszweig. Der Hauptzweig in einem Versionsverwaltungssystem hat meistens einen speziellen Namen. ( z.B in SVN->\textit{trunk} und in GIT->\textit{master})
\end{block}
\begin{block}{\textit{Objektmodell}}
Git-Objekte (blob, tree, commit, tag) sind in einer Objektdatenbank gespeichert und über SHA1-Summen identifizierbar. Die History eines Repository lässt sich als Graph von Objekten modellieren.
\end{block}
\end{frame}

%--- Grundbegriffe (GIT) ---
\begin{frame}\frametitle{Grundbegriffe (GIT)}
\begin{block}{\textit{Index}}
Der \textit{Index} ist ein lokaler Zwischenspeicher. Alle Änderungen werden zuerst in den index geschrieben. Anschießend wird der Index durch einen \textit{commit} in das Repository eingecheckt. 
\end{block}
\begin{block}{\textit{Clone}}
ist eine Kopie eines Repositories mit der gesamten History der Entwicklung.
\end{block}
\begin{block}{\textit{Tag}}
Ein Tag ist ein symbolischer Name für schwer zu merkende SHA-1 Summen. So können spezielle \textit{Commit} einen Namen, also ein \textit{Tag} erhalten. 
\end{block}
\end{frame}

% Grundlagen
\section{Grundlagen}
\begin{frame}[c]
\begin{center}
\begin{Huge}
Grundlagen und die Git Konzepte
\end{Huge}
\begin{itemize}
\vspace{1.5cm}
\item History
\item Lokale, zentrale und verteilte Versionsverwaltungssysteme
\item Zentral VS Dezentral
\item Der Index
\item Objektmodel in Git
\end{itemize}
\end{center}
\end{frame}


%----- Lokale, zentrale und verteilte Versionsverwaltungssysteme -----
\subsection{Lokale, zentrale und verteilte Versionsverwaltungssysteme} 

%----- Branches, Merges und Tags -----
\begin{frame}\frametitle{Lokale, zentrale und verteilte Versionsverwaltungssysteme}
\begin{description}
\item[lokal] rcs: einfache Ergänzung des Dateisystems.
\item[zentral] Subversion (SVN), CVS: Kommunikation nur über zentralen Server.
\item[verteilt] Git, Baszar, Merkurial: Kommunikation beliebig.

\end{description}

\begin{figure}
\includegraphics[scale=.15]{Bilder/VCSArten}
\caption{Unterschiedliche Arten von Versionsverwaltungssystemen
 \footnote{Quelle: \href{http://progit.org/}{Pro Git - Scott Chacon}}}
\end{figure}
\end{frame}

%----- Zentral VS Dezentral -----
\subsection{Zentral VS Dezentral}

%----- Die Kathedrale und der Basar -----
\begin{frame}\frametitle{Die Kathedrale und der Basar}
\begin{columns}
        \column{.55\textwidth}
 \begin{scriptsize}        
        Der legendäre Artikel \href{http://www.catb.org/~esr/writings/cathedral-bazaar/hacker-history/hacker-history.ps}{\textit{The Cathedral and the Bazaar (1997)}} von Eric S. Raymond vergleicht zwei verschiedene Entwicklungskonzepte für Software. 
        
     
  \begin{itemize}
        \item \textbf{Basar}: Ein großer Marktplatz voller Entwickler, keine vorwiegende Struktur erkennbar.
        \item \textbf{Kathedrale}: Entwickler folgen der geistigen Führung in einer sehr hierachischen Struktur. 
		\item Linus Torvalds nutzte als erster die Basar Struktur zur Entwicklung des Linux Kernels
		\item Ein Beispiel für eine Kathedrale ist die Struktur der Windows Entwicklung		
		\item es gibt auch gemischte Strukturen , z.B. wird der MATLAB Core in einer Kathedrale und die Toolboxen auf dem Basar entwickelt 
		\item ESR war über den extrem schnellen Erfolg des Basars verwundert. Lag der Erfolg an der Struktur oder Linus, ...?
		\item ESR zeigt anhand von \textit{fetchmail} (früher: \textit{popclient}) die Vorteile von einem Basar auf, siehe Artikel!
\end{itemize}
\end{scriptsize}
    
        \column{.45\textwidth}
\begin{figure} 
\includegraphics[scale=.45]{Bilder/esr.jpg}
\caption{{\scriptsize Eric S. Raymond (ESR) ist ein US-amerikanischer Autor und Programmierer in der Open-Source-Szene}}
\end{figure}

\end{columns}
\end{frame}

%----- Zentral VS Dezentral -----
\begin{frame}\frametitle{Zentral VS Dezentral}
\begin{figure}
\includegraphics[scale=.4]{Bilder/centralVSdecentral}
\end{figure}
\begin{columns}
        \column{.35\textwidth}
        \textbf{Zentrales VCS}
\begin{tiny}
\begin{itemize}
\item Preformance und Skalierbarkeit hängig von vom Server ab
\item Hohe Serverbelastung
\item Single Point of Failure
\item Backups sehr Wichtig!
\item Hoher Administationsaufwand
\item Verbindung zu Server notwendig um VCS nutzen zu können.
\end{itemize}
\end{tiny}
        \column{.65\textwidth}
        \textbf{Dezentrales VCS}
\begin{tiny}
        \begin{itemize}
\item deutlich geringer Serverbelastung beim \textit{zentralen} Speicher
\item jeder Entwickler hat die Komplette Versionsgeschichte local
\item Administationsaufwand je nach verwendeter Architektur deutlich geringer
\item \textbf{keine }Verbindung zu Server notwendig um VCS nutzen zu können.
\item kann z.B. mit der Portablen Version von Git sogar ohnen "Admin" Rechte installiert und verwendet werden.
\end{itemize}
\end{tiny}
\end{columns}
\end{frame}

%----- History -----
\subsection{Histroy}

%--- History ---
\begin{frame}\frametitle{History}
\begin{columns}
        \column{.5\textwidth}
Nach jeder abgeschlossenen Änderung werden diese in das VCS übertragen. Das VCS verbindet diese Commits miteinander in Form eines Graphen. Es wird ein gerichteter azyklischer Graph aufgebaut (Directed Acyclic Graph, DAG). 
        \column{.5\textwidth}
		\begin{figure}
		\includegraphics[scale=0.30]{Bilder/graph} 
%		\caption{Branches, Merges und Tags \footnote{Quelle:	\href{http://en.wikipedia.org/wiki/File:Subversion_project_visualization.svg}{Wikipedia}}}
\end{figure}
\end{columns}

\begin{block}{\textit{Graph}}
Ein \textit{Graph} besteht aus den beiden Kernelementen \textit{Knoten} und \textit{Kante}. Ein Commit, Tag, Referenz oder andere Objekte werden immer durch einen Knoten im Graphen dargestellt. Die Kante wird durch einen Verweis auf ein oder mehrere Eltern-Objekt(e) dargestellt. 
\end{block}

In Git hat jedes Objekt einen eindeutigen SHA1-Prüfsummen. Daraus ergibt sich eine kryptographisch gesicherte Integrität des Repositorys. 

\end{frame}

%--- History als Graph: Von Branches, Merges und Tags ---
\begin{frame}\frametitle{History als Graph: Von Branches, Merges und Tags}
\begin{description}
\item[Branch] ensteht wenn mehrere Versionen von einem Elternobjekt abhängen.
\item[Tag] ist der Name für eine bestimmte Version. 
\item[Merge] führt den Parallelen Zweig in den Hauptzweig zurück.
\item[master] ist in Git der Name für den Hauptentwicklungszweig.
\end{description}

\begin{figure}
\includegraphics[scale=0.15]{2000px-Subversion_project_visualization} 
{\tiny \caption{Der Graph bildet das gesamte Repository (Branches, Merges und Tags) ab. In Subversion: master = trunk.}}
\end{figure}
\end{frame}

%--- Objekte in Git ---
\begin{frame}\frametitle{Objekte in Git}
\begin{description}
\item[Blob] Die eigentliche Datei, SHA1-Wert und einige Metadaten.
\item[Tree] ist eine Sammlung von Blobs. Ein Tree ist damit äquivalent zu einem Verzeichnisordner
\item[Commit] speichert die Commitdaten, User, Datum, Beschreibung und verweise auf Tree, Blob und Tag ab.
\item[Tag] verbindet die SHA1-Summe eines anderen Objektes mit einem beliebigen Namen. 
\end{description}

\begin{figure}
\includegraphics[scale=0.35]{Bilder/objects-example} 
{\tiny \caption{Die verschiedenen Objekte (Commit, Tree und Blob) in Git}}
\end{figure}
\end{frame}


%----- Tools -----
\section{Tools}
\begin{frame}[c]
\begin{center}
\begin{Huge}
GUI-Tools für Git
\end{Huge}

gitk, git-gui, meld, TortoiseGit
\end{center}
\end{frame}


%----- gitk -----
\subsection{gitk}

%--- Repository Tool: gitk ---
\begin{frame}\frametitle{Repository Tool: \textit{gitk}}
\begin{block}{\textit{gitk}}
In Tcl programmiertes grafisches Frontend zur Anzeige des Repositories. Ist im Git Standard Umfang enthalten und somit \textbf{immer} vorhanden. Ermöglicht einen schnellen Überblick über die History, Commits, Diffs, Tags und die Struktur des Repositories. 
\end{block}

\begin{figure}
\includegraphics[scale=0.7]{Bilder/gitk} 
\caption{\textit{gitk}: einfach, übersichtlich und immer da!}
\end{figure}
\end{frame}

\subsection{git-gui}
\begin{frame}\frametitle{Commit Tool: \textit{git-gui}}
\begin{block}{\textit{git-gui}}
Einfaches grafisches Programm um Änderungen bereitzustellen und einen Commits zu erstellen. 
\end{block}

\begin{figure}
\includegraphics[scale=1.1]{Bilder/git_gui} 
\caption{\textit{git-gui}: Frontend für die Erstellung eines Commits. Ist wie \textit{gitk} im Standardumfang enthalten.}
\end{figure}
\end{frame}

%----- diff-Tools -----
\subsection{diff-Tools}

%--- Diff-Tool: meld ---
\begin{frame}\frametitle{Diff-Tool: meld}
%\begin{block}{\textit{gitk}}
%Diffs betrachten und mergen (zusammenführen) von Änderungen und Konflikten.
%\end{block}

\begin{figure}
\includegraphics[scale=0.9]{Bilder/meld} 
\caption{\textit{meld}: einfach, übersichtliches diff-Tool}
\end{figure}
\end{frame}

%----- TortoiseGit -----
\subsection{TortoiseGit}


%--- tortoisegit ---
\begin{frame}\frametitle{TortoiseGit}
\textbf{TortoiseGit} ist ein kostenloser und freier Windows Client für Git.

\begin{figure}
\includegraphics[scale=0.25]{Bilder/TortoiseGitIcon4} 
\caption{\textit{tortoiseGit}: Git Integration in den Windows Explorer}
\end{figure}
\end{frame}

%--- tortoisegit 2 ---
\begin{frame}\frametitle{TortoiseGit 2}
GIT-Befehle in das Kontext Menü im Windows Explorer integriert:

\begin{figure}
\includegraphics[scale=0.5]{Bilder/TortoiseGita} 
\caption{Git Befehle im Kontextmenü des Explorers. Es werden nur die nutzbaren Befehle angezeigt.}
\end{figure}
\end{frame}

%--- tortoisegit 3 ---
\begin{frame}\frametitle{TortoiseGit 3}
Als erstes sollte man seine Benutzerdaten eintragen:

\begin{figure}
\includegraphics[scale=0.25]{Bilder/TortoiseGitConfig} 
\caption{Unter \textbf{Settings} kann man sollte man Benutzer und Email einstellen!}
\end{figure}
\end{frame}

%--- tortoisegit 4 ---
\begin{frame}\frametitle{TortoiseGit 4}
Mit den \textbf{Commit...} Button kann man ganz einfach den aktuellen Stand sichern:

\begin{figure}
\includegraphics[scale=0.35]{Bilder/TortoiseGitCommit} 
\caption{TortoiseGit bietet die Möglichkeit nur die ausgewählten Dateien in einen Commit zusammenzufassen!}
\end{figure}
\end{frame}


\begin{frame}\frametitle{TortoiseGit 5}

\begin{columns}
        \column{.45\textwidth}
        Merkmale von TortoiseGit:
                \begin{enumerate}
                \item Intuitiv und schnell zu erlernen
                \item gute Integration in Windows durch Overlays
                \item add, commit und push mit einem Dialog
                \item Unterstützt gängige Speicherorte, u.a. Netzlaufwerke, GitHub, ...
                \end{enumerate}
        \column{.55\textwidth}
                \pgfimage[width=\textwidth]{Bilder/TortoiseGit2.png}                
\end{columns}
\end{frame}

%----- Befehle -----
\section{Befehle}

%--- Befehle ---
\begin{frame}[fragile]\frametitle{Befehle}
\begin{block} {Getting Started}
\begin{tiny}
Als erstes sollte man Git seine Identiät bekannt machen:
\begin{lstlisting}
git config --global user.name "Bernd Wunder"
git config --global user.email bernd.wunder@leb...
\end{lstlisting}

neues Repository im aktuellen Verzeichnis anlegen:
\begin{lstlisting}
git init
\end{lstlisting}
%\lstinputlisting{new.txt}

eine Datei dem Repository hinzufügen:
\begin{lstlisting}
git add /Pfad/Zur/Datei
\end{lstlisting}

alle Dateien im aktuellen Verzeichnis dem aktuellen Repository hinzufügen:
\begin{lstlisting}
git add .
\end{lstlisting}

einen Commit ausführen:
\begin{lstlisting}
git commit -m "Initial Commit"
\end{lstlisting}
\end{tiny}
\end{block}
\end{frame}

%--- Befehle 2---
\begin{frame}[fragile]\frametitle{Befehle 2}
\begin{block} {Remotes}
Unter einem Remote versteht man eine entfernte Quelle. Der \textbf{\lstinline|git remote|} Befehl dient zum Verwalten der Remotes:
\begin{lstlisting}[mathescape=true]
bernd@power:~$\$$ $\textbf{git remote}$
origin
power
\end{lstlisting}
%

oder ausführlicher mit:
\begin{lstlisting}[mathescape=true]
bernd@power:~$\$$ $\textbf{git remote -v}$
origin  /media/Transcend/Versionsverwaltung_mit_GIT/ (fetch)
origin  /media/Transcend/Versionsverwaltung_mit_GIT/ (push)
\end{lstlisting}
%

einen neuen Alias auf einen entfernten Remote hinzufügen:
\begin{lstlisting}
git remote add newName https://www.weitWeg.de/Repository.git
\end{lstlisting}
\end{block}
\end{frame}

%--- Befehle 3 ---
\begin{frame}[fragile]\frametitle{Befehle 3}
\begin{block} {branch}
Für die Erstellung eines neuen Zweiges (branch):
\begin{lstlisting}
git branch lilly
\end{lstlisting}

Eine ausführliche Übersicht über die vorhandenen Branches:
\begin{lstlisting}[mathescape=true]
bernd@Power:~$\$$ $\textbf{git branch -v}$
* lilly  98a74d9 Some further commands
  master e2694e7 Some explaination about commands
\end{lstlisting} %

Umbenenen des aktuellen Zweiges nach isabella:
\begin{lstlisting}
git branch -m isabella
\end{lstlisting}

In einen anderen Zweig wechseln und auschecken (hier in master):
\begin{lstlisting}
git checkout master
\end{lstlisting}

\end{block}
\end{frame}

%----- Sonstiges -----
\section{Sonstiges}
\begin{frame}[c]
\begin{center}
\begin{Huge}
Sonstiges 
\end{Huge}

Protokolle, .gitignore, Kooperation mit anderen VCS
\end{center}
\end{frame}

%----- Protokolle -----
\subsection{Protokolle}

%--- Protokolle ---
\begin{frame}[fragile]\frametitle{Protokolle}
\begin{block} {Protokolle}
Zwischen den Repositories können Daten mit einer Reihe unterschiedlicher Protokolle ausgetauscht werden:
\begin{itemize}
\item http (Webserver)
\item https (Webserver, verschlüsselt)
\item ftp (Webserver)
\item ssh (Secure Shell, verschlüsselt, admins)
\item rsync (GNU/Linux, Synchronisationsprotokoll mit Delta-Kodierung)
\item git (git-Protokoll, gepackt)
\item email (Patches via Email, Mailing-Liste)
\end{itemize}
\end{block}
\end{frame}

%--- Protokolle - Beispiele ---
\begin{frame}[fragile]\frametitle{Protokolle - Beispiele}
\begin{block} {\textit{clone}-Befehl}
Um ein entferntes Repository zu klonen:
\begin{lstlisting}
git clone <remote> <local>
\end{lstlisting}
\end{block}
\begin{exampleblock} {git-Quellcode von github.com klonen}
über das git-Protokll von github.com kopieren:
\begin{lstlisting}
git clone git://github.com/gitster/git.git git
\end{lstlisting}
und über https:
\begin{lstlisting}
git clone https://github.com/gitster/git.git git
\end{lstlisting}
oder von einer localen Quelle:
\begin{lstlisting}
git clone /home/julia/git git
\end{lstlisting}
\end{exampleblock}
\end{frame}

%----- .gitignore -----
\subsection{.gitignore}

%--- .gitignore ---
\begin{frame}[fragile]\frametitle{.gitignore}
\begin{footnotesize}
Um bestimmte Dateien oder Muster nicht unter Versionsverwaltung zu stellen kann man die \textbf{\textit{.gitignore}} Datei verwenden. Alle darin enthaltenen Dateien oder Muster werden ignoriert.

In der Regel werden bei einem Buildprozeß Hilfsdateien angelegt die nur für die Tools notwendig sind. Diese Dateien möchte man in der Regeln nicht versionieren. Auch Projektdateien und Konfigurationsdateien von der Entwicklungsumgeben haben in der Versionsverwaltung nichts zu suchen, da diese sich mit unterschiedlichen Benutzern oder Programmen ändern. 

Die \textbf{\textit{.gitignore}} Datei kann sich selbst im entsprechenden Projektordner 
befinden und unter Versionsverwaltung gestellt werden. Daneben kann man zusätzlich
eine eigene Datei die nicht im unter Versionsverwaltung steht und sich sinnvollerweise im eigenen Home
Verzeichnis befindet mit einbinden: 
\end{footnotesize}
\begin{lstlisting}
git config --global core.excludesfile ~/.gitignore
\end{lstlisting}
\end{frame}

%--- .gitignore 2 ---
\begin{frame}[fragile]\frametitle{.gitignore 2}
\begin{block} {\textbf{\textit{.gitignore}}}
Eine \textbf{\textit{.gitignore}} Datei sieht nun z.B. wie folgt aus:
\begin{lstlisting}[mathescape=true]
# do not versioning all LaTeX files!
*.log
*.out
# ...

# Verzeichnis komplett ignorieren
~/temp/

# ! invertiert den Wunsch, so wird die Datei 
# ImportendLogging.log doch von git verwaltet, 
# obwohl obiges Muster *.log zutrifft. 
!ImportendLogging.log
\end{lstlisting}
\end{block}
\end{frame}

%----- Kooperation mit anderen VCS -----
\subsection{Kooperation mit anderen VCS}

%--- Kooperation mit anderen VCS ---
\begin{frame}\frametitle{Kooperation mit anderen VCS}
Git bietet sehr gute kooperation mit anderen Systemen wie Basar, Mercurial, SVN, CVS

\begin{columns}
        \column{.5\textwidth}
                \begin{itemize}                
					\item git-svn -> besser wenn Hauptrepository in GIT
					\item subgit (Subversion (SVN) + Git) -> besser wenn Hauptrepository in SVN
					\item git-cvs	
					\item hg-git
					\item git-hg
					\item git-bzr					
					\item ...
                \end{itemize}
        \column{.5\textwidth}
\begin{figure}
\includegraphics[scale=0.4]{Bilder/subgit} 
{\footnotesize \caption{Git arbeitet gut mit anderen VCS zusammen, wie z.B. mit Subversion (SVN)}}
\end{figure}
\end{columns}
\end{frame}

%----- Submodules -----
\subsection{Submodules}

%--- Kooperation mit anderen VCS ---
\begin{frame}\frametitle{Submodules}
Unter einem Submodul versteht man ein Repository, das sich in einem Unterordner eines anderen Repositories befindet. Dies bietet sich z.B. an um eine eigene Bibliothek auf eine \textit{intelligente} Art einzubinden. 

Bindet man eine externe Bibliothek ein und führt Anpassungen durch, können folgende Probleme auftretten:
\begin{itemize}                
	\item Einbinden einer statischen festen Version: Durch die Anpassungen wird ein folgen des Mainstreams sehr aufwendig. 
	\item Bibliothek als gegeben voraussetzen: Anpassungen an der Bibliothek sind damit ausgeschlossen. 
\end{itemize}

Durch die Verwendung von Submodules in Git kann man die beiden beschriebenen Probleme umgehen.
\end{frame}

%----- Quellen ----- 
\section[Quellen]{Referezen}

%--- Quellen und Literatur ---
\begin{frame}\frametitle{Quellen \& Literatur}
\begin{thebibliography}{9}
\bibitem[a]{a} \emph{\href{http://wiki.bazaar.canonical.com/BzrVsGit}{Bazaar vs Git}}
\bibitem[a1]{a1} \emph{\href{http://de.wikipedia.org/wiki/Versionsverwaltung}{Versionsverwaltung}}
\bibitem[a2]{a2} \emph{\href{http://progit.org/}{Pro Git - Scott Chacon}}
\bibitem[a3]{a3} \emph{\href{http://better-scm.shlomifish.org/comparison/comparison.html}{Version Control System Comparison}}
\bibitem[a4]{a4} \emph{\href{http://www.madboa.com/geek/rcs/} {RCS HowTo}}
\bibitem[a5]{a5} \emph{\href{http://doc.bazaar.canonical.com/migration/en/why-switch-to-bazaar.html}{Why Switch to Bazaar?}}
\bibitem[a6]{a6} \emph{\href{http://wiki.ubuntuusers.de/Bazaar}{Bazaar}}
\bibitem[a7]{a7} \emph{\href{http://wiki.eclipse.org/EGit/User\_Guide} {EGit User Guide}}
\end{thebibliography}
\end{frame}

%--- Quellen und Literatur ---
\begin{frame}\frametitle{Quellen \& Literatur 2}
\begin{thebibliography}{9}
\bibitem[a8]{a8} \emph{\href{http://wiki.ubuntuusers.de/Git}{Git}}
\bibitem[a9]{a9} \emph{\href{https://github.com/features/projects}{Github}}
\bibitem[a10]{a10} \emph{\href{http://book.git-scm.com/index.html}{The Git Community Book}}
\bibitem[a11]{a11} \emph{\href{http://git-scm.com/}{Git Projekt Page}}
\bibitem[a12]{a12} \emph{\href{http://git.or.cz/course/svn.html}{Git - SVN Crash Course}}
\bibitem[a13]{a13} \emph{\href{http://de.whygitisbetterthanx.com/}{Warum Git besser als X ist}}
\bibitem[a14]{a14} \emph{\href{http://velian.dyndns.org/mediawiki/index.php/SVN\_und\_GIT\_im\_Vergleich}{SVN und GIT im Vergleich}}
\bibitem[a15]{a15} \emph{\href{http://www.markshuttleworth.com/archives/123}{Renaming is the killer app of distributed version control}}
\end{thebibliography}
\end{frame}

\end{document}

